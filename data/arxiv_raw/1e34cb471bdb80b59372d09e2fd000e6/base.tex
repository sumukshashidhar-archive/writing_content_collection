\usepackage{ifthen}
\usepackage{natbib}
\usepackage{amsmath,amssymb}
\usepackage{amsthm}
\usepackage{bm}
\usepackage[usenames,dvipsnames,svgnames,table]{xcolor}
\usepackage[hyperindex,
            linktocpage=true,
            colorlinks=true,
            linkcolor=blue,
            urlcolor=blue,
            citecolor=blue,
            anchorcolor=blue
            ]{hyperref}
\usepackage{stmaryrd}
            
\ifdraft
\usepackage[notref,notcite]{showkeys}
\fi

\usepackage{algorithm}
\usepackage{subcaption}
\usepackage{listings}
\renewcommand{\lstlistingname}{Algorithm}
\renewcommand{\lstlistlistingname}{List of \lstlistingname s}
\lstset{basicstyle=\ttfamily,breaklines=true,basewidth=0.5em}
\lstset{framextopmargin=1pt,frame=lines}
\newcommand{\lstmathcomment}[1]{\textcolor{darkorange}{#1}}
\usepackage{letltxmacro}
\newcommand*{\SavedLstInline}{}
\LetLtxMacro\SavedLstInline\lstinline
\DeclareRobustCommand*{\lstinline}{
  \ifmmode
    \let\SavedBGroup\bgroup
    \def\bgroup{
      \let\bgroup\SavedBGroup
      \hbox\bgroup
    }
  \fi
  \SavedLstInline
}
\newcommand{\code}[1]{\texttt{\lstinline[mathescape,classoffset=1,keywordstyle=\color{black},basicstyle=\color{black},classoffset=0,keywordstyle=\color{black}]{#1}}}
\newcommand{\lstcommentcolor}[1]{\textcolor{darkorange}{#1}}

\usepackage[capitalize]{cleveref}
\crefname{listing}{Algorithm}{Algorithms}
\Crefname{listing}{Algorithm}{Algorithms}

\usepackage{graphicx}
\usepackage{fancyhdr}


\theoremstyle{definition}

\newcommand{\mynewtheorem}[3]{
\ifthenelse{\equal{#1}{theorem}}{
    \newtheorem{my#1}{#2}
}{
    \newtheorem{my#1}[mytheorem]{#2}
    \ifthenelse{\equal{#3}{}}{
        \crefname{my#1}{#2}{#2s}
    }{
        \crefname{my#1}{#2}{#3}
    }
}
}

\mynewtheorem{theorem}{Theorem}{}
\mynewtheorem{lemma}{Lemma}{}
\mynewtheorem{corollary}{Corollary}{Corollaries}
\mynewtheorem{definition}{Definition}{}
\mynewtheorem{assumption}{Assumption}{}
\mynewtheorem{proposition}{Proposition}{}
\mynewtheorem{remark}{Remark}{}
\mynewtheorem{example}{Example}{}
\mynewtheorem{exercise}{Exercise}{}

\newcommand{\myqedsymbol}{$\diamond$}

\newenvironment{theorem}
  {\pushQED{\qed}\renewcommand{\qedsymbol}{\myqedsymbol}\mytheorem}
  {\popQED\endmytheorem}
\newenvironment{lemma}
  {\pushQED{\qed}\renewcommand{\qedsymbol}{\myqedsymbol}\mylemma}
  {\popQED\endmylemma}
\newenvironment{corollary}
  {\pushQED{\qed}\renewcommand{\qedsymbol}{\myqedsymbol}\mycorollary}
  {\popQED\endmycorollary}
\newenvironment{proposition}
  {\pushQED{\qed}\renewcommand{\qedsymbol}{\myqedsymbol}\myproposition}
  {\popQED\endmyproposition}
\newcommand{\theproposition}{\themyproposition}
\newenvironment{definition}
  {\pushQED{\qed}\renewcommand{\qedsymbol}{\myqedsymbol}\mydefinition}
  {\popQED\endmydefinition}
\newenvironment{assumption}
  {\pushQED{\qed}\renewcommand{\qedsymbol}{\myqedsymbol}\myassumption}
  {\popQED\endmyassumption}
\newenvironment{remark}
  {\pushQED{\qed}\renewcommand{\qedsymbol}{\myqedsymbol}\myremark}
  {\popQED\endmyremark}
\newenvironment{example}
  {\pushQED{\qed}\renewcommand{\qedsymbol}{\myqedsymbol}\myexample}
  {\popQED\endmyexample}
\newenvironment{exercise}
  {\pushQED{\qed}\renewcommand{\qedsymbol}{\myqedsymbol}\myexercise}
  {\popQED\endmyexercise}
  
  
 

\definecolor{darkred}{rgb}{.5,0,0}
\definecolor{darkgreen}{rgb}{0,.5,0}
\definecolor{darkblue}{rgb}{0,0,.5}
\definecolor{darkorange}{rgb}{.8,.4,0}
\ifdraft
\newcommand{\todo}[1]{\textcolor{darkorange}{(\emph{TODO: #1})}}
\newcommand{\comment}[1]{\textcolor{gray}{(\emph{#1})}}
\newcommand{\warning}[1]{\textcolor{red}{(\emph{WARNING: #1})}}
\newcommand{\quest}[1]{\textcolor{darkgreen}{(\emph{Q: #1})}}
\newcommand{\XXX}{\textcolor{red}{\textbf{XXX}}}
\newcommand{\mh}[1]{\textcolor{darkblue}{(MH: \emph{#1}})}
\else
\newcommand{\todo}[1]{}
\newcommand{\comment}[1]{}
\newcommand{\warning}[1]{}
\newcommand{\quest}[1]{}
\newcommand{\XXX}{}
\newcommand{\mh}[1]{}
\fi



\newcounter{alphoversetcount}
\newcommand{\alphnextref}{\stepcounter{alphoversetcount}\text{(\alph{alphoversetcount})}}
\newcommand{\alphoverset}[1]{\overset{\alphnextref{}}{#1}}
\newcommand{\resetalph}{\setcounter{alphoversetcount}{0}}

\newenvironment{alphalign*}{
\csname align*\endcsname\resetalph{}
}{
\csname endalign*\endcsname\resetalph{}
}

\newenvironment{alphalign}{
\csname align\endcsname\resetalph{}
}{
\csname endalign\endcsname\resetalph{}
}